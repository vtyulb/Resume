\documentclass[a4paper,10pt]{article}
\usepackage{xcolor}
\usepackage{hyperref}
\usepackage[utf8]{inputenc}
\usepackage[russian]{babel}
\definecolor{linkcolor}{HTML}{799B03} 

%opening
\title{\huge Резюме}
\author{\Huge Тюльбашев Владислав Сергеевич}

\begin{document}
\maketitle
\section{Основные достижения} {
    \subsection{Призер всероссийской олимпиады школьников по программированию 2012(23 место)}{}
    \subsection{Призер всероссийской олимпиады школьников по программированию 2013(22 место)}{}
}
\section{Публикации} {
    \subsection{\href{http://habrahabr.ru/post/185770/}{QLiveBittorrent}}{}
    \subsection{\href{http://habrahabr.ru/post/177807/}{Labyrus}}{}
}
\section{Образование} {
    \bfseries
    Общее среднее (полное) - СУНЦ МГУ.
    Сейчас поступил на ВМК (1 курс).
}
\section{Технологии} {
    \subsection{Pascal}{}
    \subsection{C, C++}{}
    \subsection{latex}
    \subsection{brainfuck}
    \subsection{Qt}{}{}
    \subsection{OpenGl}{}
    \subsection{FUSE}{}
    \subsection{libtorrent-rasterbar}{}
    \subsection{Навыки установки и администрирования операционных систем семейств Windows, Linux, в особенности ArchLinux.}{}
    \subsection{VirtualBox}{}
    \subsection{Алгоритмы}{
	\subsubsection{Динамическое программирование}{}
	\subsubsection{Графы}{Обход в ширину, поиск в глубину, Дейкстра, Флойд, Форд-Беллман,
	    \paragraph{Каркасы}{Алгоритм краскала, Алгоритм Прима}
	    \paragraph{Потоки}{Форд-Фалкерсон, Диница, масштабирование}
	}
	\subsubsection{Структуры данных}{
	    \paragraph{Дерево отрезков}{}
	    \paragraph{Дерево фенвика(в том числе многомерное)}{}
	    \paragraph{Декартого дерево}{}
	    \paragraph{Сплей-дерево}{}
	    \paragraph{Куча, очередь, стэк, дек}{}
	    \paragraph{Персистентные}{Все то же самое, но персистентное, за исключением сплей-дерева, дерева фенвика, кучи.}
	}
	\subsubsection{Прочие сложные и не очень алгоритмы}{sqrt-декомпозиция, разбор~выражений, длинная~арифметика, 
			суффиксный~массив, Z-функция, префикс-функция, поиск~мостов~и~точек~сочленения, теория~игр, 
			вычислительная~геометрия, LCA}
    }
}
\begin{figure}[b]
    Данный текст свёрстан с использованием системы \LaTeX.
\end{figure}

\end{document}
